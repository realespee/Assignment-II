\documentclass{article}
\begin{document}
\title{CALCULATING NETWORK SECURITY METRICS}
\author{CS EVE Group 206 }
\maketitle
\section{Introduction}
The calculation of network security metrics is a complex problem which involves understanding the state and configuration of network connections, devices and protocols. This research analyses the problem of calculating network security metrics and proposes a framework which can calculate security metrics for a typical small network comprising of numerous devices, operating systems and hosts. The project will involve the configuration of virtual networks for testing the framework. CVSS or other metric scores may be used to aid the calculation of the metric. This document involves some practical work which may involve setting up a virtual network which has a number of machines on the network and applying a range of scanning, probing and vulnerability testing mechanisms on the network
\section{Background to the Problem}
The need for metrics is great, because all companies suffer from security-related aches
and pains. Sometimes the pain is sharp and incapacitating, such as when an intruder
defaces a public-facing website. Perhaps, as with the now-defunct Egghead Software, an
intruder successfully obtains sensitive customer data, and the resulting embarrassment
causes business losses. Sharp pains are the kind that put companies “on the front page of
the Wall Street Journal,” as the expression goes. Far more common are the dull aches:
an unsettling feeling in the CIO’s stomach that something just isn’t right. Regardless of
the source of the pain, security metrics can help with the diagnosis.
\section{Problem Statement}
The calculation of network security metrics is a complex problem which involves understanding the state and configuration of network connections, device and protocols. Security metrics are essential to comprehensive network security and CSA management, without good metrics analysts cannot answer many security related questions

\section{Objectives}
\subsection{General Objectives}
The main objective of this project is to quantify the vulnerability of a network using CVSS and other metric scores.
\subsection{Specific Objectives}
\begin{enumerate}
\item Understand Security risks 
\item Establish emerging problems
\item Understand weaknesses in the security infrastructures
\item Measure performance of counter measure processes
\item Recommend technology improvements
\end{enumerate}

\section{Methodology}
CVSS(v3) or other metric scores may be used to aid the calculation of the metric. 
CVSS assigns scores to vulnerabilities, allowing responders to prioritize responses and resources according to the threat. CVSS assessment measures three main areas of concern:

\begin{enumerate}
\item Base Metrics for qualities intrinsic to a vulnerability
\item Modified vector. The Modified Base is intended to reflect differences within an organization or company compared to the world as a whole. New metrics to capture the importance of Confidentiality, Integrity and Availability to a specific environment are available
\item	Temporal Metrics for characteristics that evolve over the lifetime of vulnerability
\end{enumerate}
A numerical score is generated for each of these metric groups
\linebreak
\linebreak
We shall also quantify the following diagnostic metrics
\begin{enumerate}
\item ANTIVIRUS AND ANTISPAM.
Under the category of antivirus and antispyware metrics are the usual “fun facts”: the
number of distinct pieces of malware detected by antimalware software scans.
Firewall and network perimeter

\item ATTACKS.
Quantifying security “attacks” is a difficult task, but it is getting easier thanks to continuing
improvements to the accuracy of intrusion detection software and, in particular,
SEIM6 software. Security vendors like ArcSight, IBM (Micromuse), and NetForensics
attempt to identify attacks by filtering security information into three levels of criticality.

\item COVERAGE AND CONTROL.
Coverage and control metrics characterize how successful an organization is at extending
the reach of its security régime

\item PATCH MANAGEMENT.
patching is an essential part of keeping systems up-to-date and in a known
state. In other words, it is part of an overall portfolio of security controls. The degree to
which an organization keeps its infrastructure up to patch indicates the effectiveness of
its overall security program
\end{enumerate}

And many others
\section{Anticipated Outcomes} Metrics tell us
which security threat outbreaks were bad enough that automated quarantine-and-removal
processes could not contain them. Dividing the number of incidents that required
human intervention into the total number of incidents gives us a much more honest
assessment of the effectiveness of the antivirus system. \linebreak
This project will come up with CVSS number score for the Basic metrics, Modified metrics and the Temporal metric. It’ll also establish metrics for patch management, coverage and control, attacks, antivirus and antispyware and more
It will also address vulnerability management and here it will address uptime and downtime metrics, availability and reliability, system recovery, change control and many more

 \begin{thebibliography}{1}

  \bibitem{notes} Andrew Jaquith {\em Security Metrics: Replacing Fear, Uncertainity and Doubt
 ISBN 0-321-34998-9.} First Edition May, 2007.

  \bibitem{impj}  Lance Hayden {\em IT Security Metrics, A practical Framework for Measuring Security and Protecting Data ISBN: 978-0-07-171341-2} 2010:
  The McGraw Hill Companies.

  \bibitem{fo}  {\em https://en.wikipedia.org/wiki/CVSS

} 

  \end{thebibliography}

\end{document}